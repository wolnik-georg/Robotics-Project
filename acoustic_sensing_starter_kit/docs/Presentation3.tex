\documentclass{beamer}
\usetheme{Madrid}
\usecolortheme{default}

\usepackage[utf8]{inputenc}
\usepackage{booktabs}
\usepackage{array}
\usepackage{colortbl}
\usepackage{graphicx}
\usepackage{tikz}
\usetikzlibrary{positioning}
\definecolor{primaryblue}{RGB}{0,100,200}  % Blue shade for primary elements
\definecolor{successgreen}{RGB}{0,150,80}  % Green shade for success/highlights
\definecolor{errorred}{RGB}{200,50,50}     % Red shade for poor performance

\title{Geometrical Reconstruction using Acoustic Tactile Sensing}
\author{Georg Wolnik}
\institute{TU Berlin - Robotics and Biology Department}
\date{\today}

\begin{document}

% Title slide
\begin{frame}
\titlepage
\end{frame}

% ───────────────────────────────────────────────
% New Slide: Goal evolution
% ───────────────────────────────────────────────
\begin{frame}{Project Goals – Evolution of Scope}

\begin{block}{Original very ambitious goal}
    Reconstruction of \textbf{3D objects and surfaces}
\end{block}

\bigskip

\begin{block}{Adjusted goal during the project}
    Reconstruction of \textbf{2D objects / patterns} on a surface \\
    (cut-out holes: circles, stars, rectangles, hexagons, etc.)
\end{block}

\bigskip

\begin{alertblock}{Current realistic goal – given remaining time \& challenges}
    Very basic 2D reconstruction
    \begin{itemize}
        \item distinguishing only between
        \begin{itemize}
            \item contact
            \item no contact
            \item edges of \textbf{squares}
        \end{itemize}
        \item limited accuracy / proof-of-concept level
    \end{itemize}
\end{alertblock}

\end{frame}

% ───────────────────────────────────────────────
% New Slide: Planned Method Focus for Remaining Time
% ───────────────────────────────────────────────
\begin{frame}{Method Focus for Remaining Project Time}

\begin{block}{Inspired by: VibeCheck: Using Active Acoustic Tactile Sensing for Contact-Rich Manipulation}
    \small Kaidi Zhang et al. (arXiv:2504.15535, 2025)
\end{block}

\bigskip

\textbf{Core idea I want to apply:}\vspace{0.3cm}

\begin{itemize}
    \item Collect acoustic data across \structure{multiple different workspaces / surface conditions}
    \item Vary \textbf{object placement/position} relative to the finger/robot (e.g. center, offset, rotated...)
    \item $\rightarrow$ Force the model to learn the actual \textbf{contact scenario} rather than robot-specific noise/vibrations
\end{itemize}

\bigskip

\begin{alertblock}{Goal for my setup}
    Implement this data collection strategy to fix current main issue: \\
    model picks up robot/finger noise instead of true contact patterns
\end{alertblock}

\end{frame}

% ───────────────────────────────────────────────
% New Slide: Additional Data Collection Strategy
% ───────────────────────────────────────────────
\begin{frame}{Additional Planned Data Collection Strategy}

\textbf{Goal:} Better separation of \structure{contact} vs. \structure{no-contact} signals  
(under identical robot/finger configuration)

\bigskip

\begin{itemize}
    \item Collect data with the \textbf{same 3D printed object} but \alert{without any cut-out holes} \\
    $\quad\Rightarrow$ pure contact signal (full/solid surface)
    
    \bigskip
    
    \item Collect data \textbf{completely without any object} \\
    $\quad\Rightarrow$ pure no-contact / background / robot-only signal
    
    \bigskip
    
    \item $\rightarrow$ Use both to create strong, clean labels for: \\
    \quad • contact (solid object) \\
    \quad • no-contact (nothing present) \\
    \quad • and later: edges (when cut-outs are present)
\end{itemize}

\bigskip

\begin{alertblock}{Expected benefit}
    Model must learn actual \textbf{contact differences} instead of \\
    confounding with slight changes in robot pose or workspace noise
\end{alertblock}

\end{frame}
% ──────────────────────────────────────────────────────────────
% FINAL CONCLUSION – SHORT, STRONG, MEMORABLE (one slide only)
% ──────────────────────────────────────────────────────────────
% ───────────────────────────────────────────────
% Final Slide: Conclusion & Look Ahead
% ───────────────────────────────────────────────
\begin{frame}{Conclusion \& Look Ahead}

\textbf{Summary of planned focus (remaining time):}
\begin{itemize}
    \item Implement VibeCheck-style data collection: \\
      \quad multiple workspaces / object positions
    \item Add clean solid-object and no-object recordings \\
      \quad $\rightarrow$ strong contact vs. no-contact labels
\end{itemize}

\bigskip

\begin{alertblock}{Conditional next step -- only if time remains and core proof works}
    \small
    If end-to-end process succeeds and ML model reliably learns \\
    to distinguish actual contact scenarios (contact / no-contact / edges):
    \begin{itemize}
        \item Introduce more complex cut-out shape \\
          \quad (e.g. hexagon, octagon...)
    \end{itemize}
\end{alertblock}

\bigskip

\centering
\textbf{Realistic goal:} Working proof-of-concept for basic acoustic 2D contact detection
\end{frame}

\end{document}