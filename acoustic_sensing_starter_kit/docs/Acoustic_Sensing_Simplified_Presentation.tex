\documentclass{beamer}
\usetheme{Madrid}
\usecolortheme{default}

\usepackage[utf8]{inputenc}
\usepackage{graphicx}
\usepackage{booktabs}
\usepackage{array}
\usepackage{xcolor}

% Define custom colors
\definecolor{excellent}{RGB}{0,128,0}
\definecolor{good}{RGB}{255,165,0}
\definecolor{poor}{RGB}{128,0,0}
\definecolor{blue}{RGB}{0,0,255}

\title{Acoustic Sensing for Robotic Touch}
\subtitle{Can Robots "Hear" What They're Touching?}
\author{Georg Wolnik}
\institute{TU Berlin Robotics Lab}
\date{\today}

\begin{document}

% Title slide
\begin{frame}
\titlepage
\end{frame}

% Overview
\begin{frame}{The Big Question}
\begin{center}
\Large \textbf{Can acoustic signals enable robots to understand what they're touching?}
\end{center}

\vspace{1cm}

\textbf{What we did:}
\begin{itemize}
    \item Recorded 1,931 audio samples from different touch scenarios
    \item Ran 6 experiments to test acoustic sensing capabilities
    \item Found clear answers about what works and what doesn't
\end{itemize}

\vspace{0.5cm}

\textbf{Bottom line:} \textcolor{excellent}{\textbf{YES}} - but it depends on the task!
\end{frame}

% Our Test Datasets
\begin{frame}{Our Test Datasets}
\begin{table}
\centering
\caption{What We Tested}
\begin{tabular}{|l|p{4cm}|c|c|}
\hline
\textbf{Dataset} & \textbf{Scenario} & \textbf{Classes} & \textbf{Samples} \\
\hline
Batch 1 & Position detection (4 spots) & 4 & 387 \\
Batch 2 & Position validation test & 4 & 352 \\
Batch 3 & Edge vs No-edge (binary) & 2 & 298 \\
Batch 4 & Material detection (wood/metal) & 2 & 264 \\
Edge v1 & Complex edge detection & 3 & 630 \\
\hline
\textbf{Total} & \textbf{Multiple scenarios} & \textbf{2-4} & \textbf{1,931} \\
\hline
\end{tabular}
\end{table}

\vspace{0.5cm}
\textbf{Each dataset} tested with 6 different experiments to understand acoustic sensing from every angle.
\end{frame}

% Experiment Overview
\begin{frame}{Our 6 Experiments}
\begin{enumerate}
    \item \textbf{Data Quality Check} \\
    \textit{Is our audio data good enough?}
    
    \item \textbf{Visualization \& Separation} \\
    \textit{Can we see differences between classes?}
    
    \item \textbf{AI Classification} \\
    \textit{Which algorithms work best?}
    
    \item \textbf{Feature Importance} \\
    \textit{Which audio features matter most?}
    
    \item \textbf{Neural Network Analysis} \\
    \textit{What do deep learning models focus on?}
    
    \item \textbf{Physics Understanding} \\
    \textit{What's the science behind acoustic sensing?}
\end{enumerate}
\end{frame}

% Experiment 1
\begin{frame}{Experiment 1: Data Quality Check}
\textbf{Question:} Is our recorded audio data good enough for analysis?

\vspace{0.5cm}

\textbf{How we tested:}
\begin{itemize}
    \item Counted samples per class (balance check)
    \item Measured signal-to-noise ratio
    \item Extracted 38 acoustic features
    \item Checked for corruption/missing data
\end{itemize}

\vspace{0.5cm}

\begin{table}
\centering
\begin{tabular}{|l|c|c|c|}
\hline
\textbf{Batch} & \textbf{Samples} & \textbf{Balance} & \textbf{Quality} \\
\hline
Position (1\&2) & 387, 352 & 98-99\% & \textcolor{excellent}{Excellent} \\
Edge Binary & 298 & 98\% & \textcolor{excellent}{Excellent} \\
Material & 264 & 100\% & \textcolor{excellent}{Excellent} \\
Edge 3-class & 630 & 100\% & \textcolor{excellent}{Excellent} \\
\hline
\end{tabular}
\end{table}

\textbf{Result:} \textcolor{excellent}{\textbf{All data is high quality!}} SNR > 40dB, perfect feature extraction.
\end{frame}

% Experiment 2
\begin{frame}{Experiment 2: Visualization \& Class Separation}
\textbf{Question:} Can we visualize differences between touch types?

\vspace{0.5cm}

\textbf{How we tested:}
\begin{itemize}
    \item PCA: Find most important data directions
    \item t-SNE: Create 2D maps of similarity
    \item Silhouette scores: Measure class separation
\end{itemize}

\vspace{0.5cm}

\begin{table}
\centering
\begin{tabular}{|l|c|c|}
\hline
\textbf{Task} & \textbf{Silhouette Score} & \textbf{Separability} \\
\hline
Material Detection & 0.463 & \textcolor{excellent}{Excellent} \\
Edge Binary & 0.347 & \textcolor{good}{Good} \\
Position Detection & 0.184-0.191 & \textcolor{good}{Moderate} \\
Edge 3-class & -0.015 & \textcolor{poor}{Poor} \\
\hline
\end{tabular}
\end{table}

\textbf{Key Insight:} Task complexity = visualization difficulty. Material detection shows clear acoustic signatures!
\end{frame}

% Experiment 3
\begin{frame}{Experiment 3: AI Classification - The Big Test}
\textbf{Question:} Can AI algorithms learn to classify touch types from acoustic features?

\vspace{0.5cm}

\textbf{How we tested:}
\begin{itemize}
    \item 7 algorithms: Random Forest, SVM, Gradient Boosting, Neural Networks...
    \item 5-fold cross-validation for reliability
    \item Compared performance across all scenarios
\end{itemize}

\vspace{0.5cm}

\begin{table}
\centering
\begin{tabular}{|l|c|c|}
\hline
\textbf{Task} & \textbf{Best Accuracy} & \textbf{Status} \\
\hline
Material Detection & 95\% & \textcolor{excellent}{\textbf{Ready for robots!}} \\
Edge Binary & 90\% & \textcolor{excellent}{\textbf{Ready for robots!}} \\
Position Detection & 78\% & \textcolor{good}{Needs engineering} \\
Edge 3-class & 67\% & \textcolor{poor}{Challenging} \\
\hline
\end{tabular}
\end{table}

\textbf{Clear winner:} Material detection. Different materials = completely different acoustic signatures!
\end{frame}

% Classification Details
\begin{frame}{Why Some Tasks Work Better Than Others}

\textbf{Material Detection (95\% - Excellent):}
\begin{itemize}
    \item Wood vs metal = completely different acoustic resonances
    \item Even simple algorithms work perfectly
    \item Clear frequency signatures
\end{itemize}

\textbf{Edge Binary (90\% - Excellent):}
\begin{itemize}
    \item Edge contact creates sharp acoustic transients
    \item Clear temporal (timing) differences
    \item SVM algorithm works best
\end{itemize}

\textbf{Position Detection (78\% - Moderate):}
\begin{itemize}
    \item Spatial differences are subtle but detectable
    \item Needs ensemble methods (Random Forest)
    \item Consistent across validation tests
\end{itemize}

\textbf{Edge 3-class (67\% - Hard):}
\begin{itemize}
    \item Contact/Edge/No-contact states overlap acoustically
    \item Only Gradient Boosting reaches 67\%
    \item Fundamental acoustic similarity between classes
\end{itemize}
\end{frame}

% Experiment 4
\begin{frame}{Experiment 4: Which Audio Features Matter Most?}
\textbf{Question:} Which acoustic features are most important for each task?

\vspace{0.5cm}

\textbf{How we tested:}
\begin{itemize}
    \item Remove features one by one
    \item Measure performance drop when missing
    \item Rank by importance
\end{itemize}

\vspace{0.5cm}

\begin{table}
\centering
\begin{tabular}{|l|c|c|c|c|}
\hline
\textbf{Task} & \textbf{Spectral} & \textbf{Temporal} & \textbf{Statistical} & \textbf{Perceptual} \\
\hline
Material & \textbf{67\%} & 8\% & 12\% & 18\% \\
Edge Binary & 42\% & \textbf{58\%} & 15\% & 12\% \\
Position & \textbf{34\%} & 19\% & 12\% & 16\% \\
Edge 3-class & 28\% & 22\% & 20\% & 19\% \\
\hline
\end{tabular}
\end{table}

\textbf{Key Insights:}
\begin{itemize}
    \item \textbf{Material:} Frequency content dominates (spectral)
    \item \textbf{Edge:} Timing changes matter most (temporal)  
    \item \textbf{Position:} Needs balanced feature combination
    \item \textbf{Complex tasks:} Use everything available!
\end{itemize}
\end{frame}

% Experiment 5
\begin{frame}{Experiment 5: What Do Neural Networks See?}
\textbf{Question:} What do neural networks pay attention to when making decisions?

\vspace{0.5cm}

\textbf{How we tested:}
\begin{itemize}
    \item Trained neural networks on each dataset
    \item Used gradient-based saliency analysis
    \item Compared with feature ablation results
\end{itemize}

\vspace{0.5cm}

\begin{table}
\centering
\begin{tabular}{|l|c|c|c|}
\hline
\textbf{Task} & \textbf{Neural Net} & \textbf{Best Traditional} & \textbf{Insight} \\
\hline
Material & 92\% & 95\% (RF) & Any method works \\
Edge Binary & 87\% & 90\% (SVM) & Traditional sufficient \\
Position & 72\% & 78\% (RF) & Trees handle complexity better \\
Edge 3-class & 54\% & 67\% (GB) & Ensembles essential \\
\hline
\end{tabular}
\end{table}

\textbf{Key Finding:} For simple tasks, traditional ML is better. For complex tasks, neural networks reveal alternative feature combinations.
\end{frame}

% Experiment 6
\begin{frame}{Experiment 6: The Physics Behind Acoustic Sensing}
\textbf{Question:} What's physically happening when robots "hear" touch?

\vspace{0.5cm}

\textbf{How we tested:}
\begin{itemize}
    \item Extracted acoustic "impulse responses"
    \item Measured resonant frequencies and damping
    \item Analyzed physical acoustic properties
\end{itemize}

\vspace{0.5cm}

\textbf{Example: Edge Detection Physics}
\begin{table}
\centering
\begin{tabular}{|l|c|c|c|}
\hline
\textbf{Contact State} & \textbf{Duration} & \textbf{Frequency} & \textbf{Physics} \\
\hline
Contact & 56 ms & 2.0 kHz & Stable coupling \\
Edge & 42 ms & 2.8 kHz & Sharp interface \\
No Contact & 23 ms & 1.2 kHz & Diffuse scattering \\
\hline
\end{tabular}
\end{table}

\textbf{Result:} Physics-only features achieved 52\% accuracy on the hardest task. \textcolor{excellent}{\textbf{The science is real!}}
\end{frame}

% Summary
\begin{frame}{Summary: Can Acoustic Sensing Work for Robotics?}
\begin{center}
\Large \textcolor{excellent}{\textbf{YES - but it depends on what you want to detect}}
\end{center}

\vspace{0.5cm}

\textbf{Task Difficulty Ranking:}
\begin{enumerate}
    \item \textbf{Material Detection (95\%):} \textcolor{excellent}{READY FOR ROBOTS NOW}
    \item \textbf{Binary Edge Detection (90\%):} \textcolor{excellent}{READY FOR DEPLOYMENT}
    \item \textbf{Position Detection (78\%):} \textcolor{good}{NEEDS ENGINEERING}
    \item \textbf{3-Class Edge Detection (67\%):} \textcolor{poor}{RESEARCH NEEDED}
\end{enumerate}

\vspace{0.5cm}

\textbf{What makes it work:}
\begin{itemize}
    \item Different materials = different acoustic resonances
    \item Contact geometry affects acoustic coupling
    \item Edge interfaces create sharp acoustic transients
    \item The physics is measurable and predictable
\end{itemize}
\end{frame}

% Practical Guidelines
\begin{frame}{Practical Implementation Guidelines}

\textbf{For Robot Developers:}

\textcolor{excellent}{\textbf{Ready for Deployment:}}
\begin{itemize}
    \item \textbf{Material discrimination systems} - Use spectral features, any algorithm works
    \item \textbf{Binary edge detection} - Focus on temporal dynamics, use SVM
\end{itemize}

\vspace{0.5cm}

\textcolor{good}{\textbf{Feasible with Engineering:}}
\begin{itemize}
    \item \textbf{Position detection} - Use ensemble methods, expect 78\% accuracy
    \item \textbf{Simple spatial discrimination} - Requires balanced feature sets
\end{itemize}

\vspace{0.5cm}

\textcolor{poor}{\textbf{Research Needed:}}
\begin{itemize}
    \item \textbf{Complex multi-class problems} - Need advanced ensemble methods
    \item \textbf{Noise robustness} - Real-world environment testing
    \item \textbf{Real-time optimization} - Processing speed improvements
\end{itemize}
\end{frame}

% Technical Recommendations
\begin{frame}{Technical Implementation Recommendations}

\begin{table}
\centering
\caption{Algorithm \& Feature Recommendations}
\begin{tabular}{|l|l|l|}
\hline
\textbf{Task} & \textbf{Best Algorithm} & \textbf{Key Features} \\
\hline
Material Detection & Random Forest & Spectral contrast, frequency content \\
Edge Binary & SVM (RBF) & Zero-crossing rate, temporal dynamics \\
Position Detection & Random Forest & Spectral centroid, bandwidth \\
Complex Tasks & Gradient Boosting & All 38 features combined \\
\hline
\end{tabular}
\end{table}

\vspace{0.5cm}

\textbf{Processing Requirements:}
\begin{itemize}
    \item \textbf{Material detection:} <10ms latency possible
    \item \textbf{Edge detection:} <15ms latency achievable
    \item \textbf{Complex tasks:} <35ms with optimization
\end{itemize}

\vspace{0.5cm}

\textbf{Implementation tip:} Start with material detection - it's the most reliable and easiest to implement!
\end{frame}

% Future Directions
\begin{frame}{Future Research Directions}

\textbf{Immediate Opportunities:}
\begin{itemize}
    \item \textbf{Multi-modal fusion:} Combine acoustic + visual + tactile
    \item \textbf{Real-world validation:} Test in noisy environments
    \item \textbf{Online learning:} Adaptive systems that improve with use
\end{itemize}

\vspace{0.5cm}

\textbf{Advanced Research:}
\begin{itemize}
    \item \textbf{Dynamic contact analysis:} Track contact changes over time
    \item \textbf{Physics-informed features:} Better use of acoustic science
    \item \textbf{Cross-platform generalization:} Work across different robots
\end{itemize}

\vspace{0.5cm}

\textbf{Applications Ready Today:}
\begin{itemize}
    \item Quality control (material verification)
    \item Grasping assistance (edge detection)
    \item Safety systems (contact monitoring)
\end{itemize}
\end{frame}

% Final Assessment
\begin{frame}{Final Assessment}
\begin{center}
\Large \textbf{Acoustic sensing is a practical and reliable technology for robotic touch discrimination}
\end{center}

\vspace{1cm}

\textbf{Key Achievements:}
\begin{itemize}
    \item \textcolor{excellent}{\textbf{95\% accuracy}} for material detection - better than many vision systems
    \item \textcolor{excellent}{\textbf{90\% accuracy}} for binary edge detection - useful for manipulation
    \item \textbf{Clear understanding} of what works, what doesn't, and why
    \item \textbf{Practical implementation guidelines} for real-world deployment
\end{itemize}

\vspace{0.5cm}

\textbf{The technology is ready for specific applications today, with clear paths for expanding capabilities.}

\vspace{0.5cm}

\begin{center}
\textbf{Questions?}
\end{center}
\end{frame}

\end{document}