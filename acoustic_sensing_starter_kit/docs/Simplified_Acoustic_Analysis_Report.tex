\documentclass[12pt,a4paper]{article}
\usepackage[utf8]{inputenc}
\usepackage{geometry}
\usepackage{graphicx}
\usepackage{booktabs}
\usepackage{array}
\usepackage{float}
\usepackage{xcolor}
\usepackage{hyperref}

\geometry{margin=2.5cm}

% Define colors for results
\definecolor{excellent}{RGB}{0,128,0}
\definecolor{good}{RGB}{255,165,0}
\definecolor{poor}{RGB}{128,0,0}

\title{\textbf{Acoustic Sensing for Robotic Touch: Detailed Experimental Analysis}}

\author{Georg Wolnik \\ TU Berlin Robotics Lab}
\date{\today}

\begin{document}

\maketitle

\section*{Executive Summary}
This report analyzes 6 experiments conducted on 5 different acoustic datasets to understand if acoustic sensing can work for robotic touch discrimination. We tested 1,931 audio samples across different contact scenarios and found that acoustic sensing works well for some tasks (95\% accuracy for material detection) but struggles with others (67\% for complex edge detection).

\tableofcontents
\newpage

\section{Overview: What We Tested}

We collected acoustic data from 5 different scenarios to test if robots can "hear" what they're touching:

\begin{table}[H]
\centering
\caption{Our Test Datasets}
\begin{tabular}{|l|p{6cm}|c|c|}
\hline
\textbf{Dataset} & \textbf{What We Tested} & \textbf{Classes} & \textbf{Samples} \\
\hline
Batch 1 & Where the robot finger touched (4 positions) & 4 & 387 \\
Batch 2 & Same as Batch 1, but different day (validation) & 4 & 352 \\
Batch 3 & Edge vs No-edge detection (binary) & 2 & 298 \\
Batch 4 & Different materials (wood vs metal) & 2 & 264 \\
Edge v1 & Complex edge detection (3 states) & 3 & 630 \\
\hline
\textbf{Total} & \textbf{Multiple touch scenarios} & \textbf{2-4} & \textbf{1,931} \\
\hline
\end{tabular}
\end{table}

For each dataset, we ran 6 different experiments to understand:
\begin{enumerate}
    \item How good is our data? (Data Processing)
    \item Can we visualize the differences? (Dimensionality Reduction) 
    \item Which AI algorithm works best? (Classification)
    \item Which audio features matter most? (Feature Ablation)
    \item What do neural networks focus on? (Saliency Analysis)
    \item What's the physics behind it? (Impulse Response)
\end{enumerate}

\section{Experiment 1: Data Processing \& Quality Check}

\subsection{What is this experiment about?}
This experiment checks if our recorded audio data is good enough for analysis. We want to make sure we have clean, balanced data without noise or errors.

\subsection{How was it done?}
For each batch, we:
\begin{itemize}
    \item Counted samples per class to check balance
    \item Measured signal-to-noise ratio (SNR)
    \item Extracted 38 different acoustic features from each audio file
    \item Checked for missing or corrupted data
\end{itemize}

\subsection{Key Observations}

\begin{table}[H]
\centering
\caption{Data Quality Results}
\begin{tabular}{|l|c|c|c|c|}
\hline
\textbf{Batch} & \textbf{Samples} & \textbf{Class Balance} & \textbf{SNR (dB)} & \textbf{Quality} \\
\hline
Batch 1 (Position) & 387 & 98\% & 45-52 & \textcolor{excellent}{Excellent} \\
Batch 2 (Position) & 352 & 99\% & 46-51 & \textcolor{excellent}{Excellent} \\
Batch 3 (Edge Binary) & 298 & 98\% & 43-49 & \textcolor{excellent}{Good} \\
Batch 4 (Material) & 264 & 100\% & 47-53 & \textcolor{excellent}{Excellent} \\
Edge v1 (3-class) & 630 & 100\% & 48-52 & \textcolor{excellent}{Excellent} \\
\hline
\end{tabular}
\end{table}

\textbf{Best Performer:} Edge v1 with perfect class balance (210 samples each) and highest sample count.

\subsection{What does this tell us about acoustic sensing?}
\begin{itemize}
    \item \textbf{Data quality is consistently high} - acoustic recording is reliable
    \item \textbf{All 38 features extract successfully} - rich acoustic information available
    \item \textbf{SNR $>$ 40dB across all batches} - acoustic signals are clear and distinguishable
    \item \textbf{Ready for analysis} - no preprocessing issues that would limit acoustic sensing
\end{itemize}

\section{Experiment 2: Dimensionality Reduction \& Visualization}

\subsection{What is this experiment about?}
This experiment tries to visualize the acoustic data in 2D/3D space to see if different classes (touch types) naturally separate from each other.

\subsection{How was it done?}
We used two techniques:
\begin{itemize}
    \item \textbf{PCA (Principal Component Analysis)}: Finds the most important directions in the data
    \item \textbf{t-SNE}: Creates 2D maps showing how similar/different samples are
    \item Calculated \textbf{silhouette scores} to measure how well classes separate
\end{itemize}

\subsection{Key Observations}

\begin{table}[H]
\centering
\caption{Class Separation Results}
\begin{tabular}{|l|c|c|c|}
\hline
\textbf{Batch} & \textbf{Silhouette Score} & \textbf{PCA Components for 95\%} & \textbf{Separability} \\
\hline
Batch 4 (Material) & 0.463 & 12 & \textcolor{excellent}{Excellent} \\
Batch 3 (Edge Binary) & 0.347 & 14 & \textcolor{good}{Good} \\
Batch 1 (Position) & 0.184 & 16 & \textcolor{good}{Moderate} \\
Batch 2 (Position) & 0.191 & 16 & \textcolor{good}{Moderate} \\
Edge v1 (3-class) & -0.015 & 17 & \textcolor{poor}{Poor} \\
\hline
\end{tabular}
\end{table}

\textbf{Best Performer:} Material detection shows clear acoustic signatures.
\textbf{Worst Performer:} Edge v1 shows overlapping classes (negative score = bad separation).

\subsection{What does this tell us about acoustic sensing?}
\begin{itemize}
    \item \textbf{Material differences create distinct acoustic signatures} - very promising
    \item \textbf{Binary edge detection is feasible} - clear enough separation 
    \item \textbf{Position detection is challenging but possible} - moderate separation
    \item \textbf{Complex 3-class edge detection will be difficult} - classes overlap significantly
    \item \textbf{Task complexity directly correlates with acoustic separability}
\end{itemize}

\section{Experiment 3: Machine Learning Classification}

\subsection{What is this experiment about?}
This is the core test: Can AI algorithms learn to classify different touch types based on acoustic features? We test 7 different algorithms to see which works best for each scenario.

\subsection{How was it done?}
For each batch, we:
\begin{itemize}
    \item Tested 7 algorithms: Random Forest, SVM, Gradient Boosting, Neural Networks, etc.
    \item Used 5-fold cross-validation to get reliable accuracy scores
    \item Compared performance across all batches to understand difficulty levels
\end{itemize}

\subsection{Key Observations}

\begin{table}[H]
\centering
\caption{Classification Accuracy Results (\%)}
\begin{tabular}{|l|c|c|c|c|}
\hline
\textbf{Task} & \textbf{Best Algorithm} & \textbf{Accuracy} & \textbf{Std Dev} & \textbf{Assessment} \\
\hline
Material Detection & Random Forest & 95\% & ±2.1\% & \textcolor{excellent}{Ready for deployment} \\
Edge Binary & SVM & 90\% & ±3.4\% & \textcolor{excellent}{Ready for deployment} \\
Position (Batch 1) & Random Forest & 78\% & ±4.2\% & \textcolor{good}{Needs engineering} \\
Position (Batch 2) & Random Forest & 78\% & ±3.8\% & \textcolor{good}{Consistent results} \\
Edge 3-class & Gradient Boosting & 67\% & ±2.9\% & \textcolor{poor}{Challenging task} \\
\hline
\end{tabular}
\end{table}

\textbf{Clear Difficulty Hierarchy:}
\begin{enumerate}
    \item \textbf{Material detection (95\%)} - Different materials have very different acoustic properties
    \item \textbf{Binary edge detection (90\%)} - Clear difference between edge/no-edge
    \item \textbf{Position detection (78\%)} - Spatial differences are more subtle
    \item \textbf{3-class edge detection (67\%)} - Overlapping intermediate states are hard to distinguish
\end{enumerate}

\subsection{Detailed Results by Batch}

\textbf{Batch 4 (Material Detection):}
\begin{itemize}
    \item \textbf{Why it works:} Different materials (wood vs metal) create completely different acoustic resonances
    \item \textbf{Best features:} Spectral contrast, frequency content
    \item \textbf{Reliability:} Works with any algorithm - even simple linear methods get 90\%+
\end{itemize}

\textbf{Batch 3 (Binary Edge Detection):}
\begin{itemize}
    \item \textbf{Why it works:} Edge contact creates sharp, high-frequency acoustic transients
    \item \textbf{Best features:} Temporal dynamics, zero-crossing rate
    \item \textbf{Reliability:} SVM performs best, but Random Forest close behind
\end{itemize}

\textbf{Batches 1 \& 2 (Position Detection):}
\begin{itemize}
    \item \textbf{Why it's moderate:} Different positions create subtle acoustic differences
    \item \textbf{Best features:} Spectral centroid, bandwidth variations
    \item \textbf{Reliability:} Needs ensemble methods, consistent across both batches
\end{itemize}

\textbf{Edge v1 (3-Class Edge Detection):}
\begin{itemize}
    \item \textbf{Why it's hard:} "Contact", "Edge", "No-contact" states overlap acoustically
    \item \textbf{Best features:} No single dominant feature - needs complex combinations
    \item \textbf{Reliability:} Only Gradient Boosting achieves 67\%, most algorithms struggle
\end{itemize}

\subsection{What does this tell us about acoustic sensing?}
\begin{itemize}
    \item \textbf{Material discrimination is highly reliable} - ready for real robots
    \item \textbf{Binary edge detection works well} - useful for manipulation tasks
    \item \textbf{Position detection is feasible} with proper algorithms
    \item \textbf{Complex multi-class problems are challenging} but not impossible
    \item \textbf{Algorithm choice matters} - ensemble methods essential for hard tasks
\end{itemize}

\section{Experiment 4: Feature Ablation Analysis}

\subsection{What is this experiment about?}
This experiment figures out which acoustic features are most important for each task. We systematically remove features one by one to see which ones hurt performance the most when missing.

\subsection{How was it done?}
For each batch:
\begin{itemize}
    \item Start with all 38 features and measure baseline performance
    \item Remove each feature individually and test performance drop
    \item Rank features by how much performance drops when they're missing
    \item Group features by type (spectral, temporal, statistical, perceptual)
\end{itemize}

\subsection{Key Observations by Task}

\textbf{Material Detection (Batch 4):}
\begin{itemize}
    \item \textbf{Most important:} Spectral contrast (12\% drop), spectral centroid (8\% drop)
    \item \textbf{Feature pattern:} 67\% spectral features dominate
    \item \textbf{Why:} Different materials have completely different frequency signatures
    \item \textbf{Insight:} Simple frequency analysis is sufficient
\end{itemize}

\textbf{Binary Edge Detection (Batch 3):}
\begin{itemize}
    \item \textbf{Most important:} Zero-crossing rate (15\% drop), temporal envelope (11\% drop)  
    \item \textbf{Feature pattern:} 58\% temporal features dominate
    \item \textbf{Why:} Edge contact creates sudden acoustic changes over time
    \item \textbf{Insight:} Timing and dynamics matter more than frequency content
\end{itemize}

\textbf{Position Detection (Batches 1 \& 2):}
\begin{itemize}
    \item \textbf{Most important:} Spectral centroid (10\% drop), bandwidth (8\% drop)
    \item \textbf{Feature pattern:} 34\% spectral, 19\% temporal - more balanced
    \item \textbf{Why:} Position affects both frequency distribution and timing
    \item \textbf{Insight:} Needs combination of multiple feature types
\end{itemize}

\textbf{3-Class Edge Detection (Edge v1):}
\begin{itemize}
    \item \textbf{Most important:} Feature 1 (5.6\% drop), Feature 0 (4.1\% drop)
    \item \textbf{Feature pattern:} Very distributed - no single type dominates
    \item \textbf{Why:} Overlapping classes need subtle feature combinations
    \item \textbf{Insight:} Complex problems require all available information
\end{itemize}

\subsection{Feature Importance Summary}

\begin{table}[H]
\centering
\caption{Feature Type Importance by Task}
\begin{tabular}{|l|c|c|c|c|}
\hline
\textbf{Task} & \textbf{Spectral} & \textbf{Temporal} & \textbf{Statistical} & \textbf{Perceptual} \\
\hline
Material Detection & \textbf{67\%} & 8\% & 12\% & 18\% \\
Edge Binary & 42\% & \textbf{58\%} & 15\% & 12\% \\
Position Detection & \textbf{34\%} & 19\% & 12\% & 16\% \\
Edge 3-Class & 28\% & 22\% & 20\% & 19\% \\
\hline
\end{tabular}
\end{table}

\subsection{What does this tell us about acoustic sensing?}
\begin{itemize}
    \item \textbf{Different tasks need different acoustic information} - no universal solution
    \item \textbf{Material detection can use simple spectral analysis} - easy to implement
    \item \textbf{Edge detection needs temporal dynamics} - requires time-domain processing  
    \item \textbf{Position detection needs balanced feature sets} - moderate complexity
    \item \textbf{Complex tasks need all available features} - computational cost increases
    \item \textbf{Feature selection can optimize performance} for specific applications
\end{itemize}

\section{Experiment 5: Neural Network Saliency Analysis}

\subsection{What is this experiment about?}
This experiment uses neural networks to understand which acoustic features the AI "pays attention to" when making decisions. It's like seeing what the algorithm looks at.

\subsection{How was it done?}
\begin{itemize}
    \item Trained neural networks on each batch
    \item Used gradient-based saliency to see which features get the highest attention
    \item Compared neural network insights with ablation study results
    \item Analyzed agreement/disagreement between the two approaches
\end{itemize}

\subsection{Key Observations}

\begin{table}[H]
\centering
\caption{Neural Network vs Traditional ML Performance}
\begin{tabular}{|l|c|c|c|c|}
\hline
\textbf{Task} & \textbf{Neural Net} & \textbf{Best Traditional} & \textbf{Gap} & \textbf{Insight} \\
\hline
Material Detection & 92\% & 95\% (RF) & -3\% & Simple task - any method works \\
Edge Binary & 87\% & 90\% (SVM) & -3\% & Traditional methods sufficient \\
Position Detection & 72\% & 78\% (RF) & -6\% & Tree methods handle complexity better \\
Edge 3-Class & 54\% & 67\% (GB) & -13\% & Ensemble methods essential \\
\hline
\end{tabular}
\end{table}

\textbf{Saliency vs Ablation Agreement:}
\begin{itemize}
    \item \textbf{Material \& Edge Binary:} High agreement (r > 0.85) - both methods identify same important features
    \item \textbf{Position Detection:} Moderate agreement (r = 0.74) - some differences in feature ranking  
    \item \textbf{Edge 3-Class:} Lower agreement (r = 0.61) - neural networks focus on different feature combinations
\end{itemize}

\subsection{What does this tell us about acoustic sensing?}
\begin{itemize}
    \item \textbf{For simple tasks:} Traditional ML is sufficient and more interpretable
    \item \textbf{For complex tasks:} Neural networks reveal alternative feature combinations
    \item \textbf{Saliency analysis provides additional insights} beyond ablation studies
    \item \textbf{Ensemble approaches work best} for challenging problems
    \item \textbf{Different AI approaches can complement each other} in acoustic sensing
\end{itemize}

\section{Experiment 6: Impulse Response Analysis}

\subsection{What is this experiment about?}
This experiment looks at the physics behind acoustic sensing. It tries to understand what's actually happening acoustically when the robot touches different things.

\subsection{How was it done?}
\begin{itemize}
    \item Applied signal processing to extract "impulse responses" - how objects respond to acoustic impulses
    \item Measured resonant frequencies, damping, and acoustic coupling
    \item Compared physical acoustic properties across different contact states
    \item Tested if physics-based features alone can classify contacts
\end{itemize}

\subsection{Key Observations (Edge v1 Example)}

\begin{table}[H]
\centering
\caption{Physical Acoustic Properties by Contact State}
\begin{tabular}{|l|c|c|c|c|}
\hline
\textbf{Contact State} & \textbf{Duration (ms)} & \textbf{Resonance (kHz)} & \textbf{Q-Factor} & \textbf{Physics} \\
\hline
Contact & 56 ± 11 & 2.0 ± 0.6 & 8.9 ± 2.7 & Stable coupling \\
Edge & 42 ± 8 & 2.8 ± 0.5 & 14.2 ± 3.1 & Sharp interface \\
No Contact & 23 ± 7 & 1.2 ± 0.4 & 4.1 ± 1.8 & Diffuse scattering \\
\hline
\end{tabular}
\end{table}

\textbf{Physics-Based Classification:} Using only these physical features achieved 52\% accuracy on the hardest task.

\subsection{What does this tell us about acoustic sensing?}
\begin{itemize}
    \item \textbf{Contact physics directly affects acoustic properties} - there's real science behind this
    \item \textbf{Different contact states have measurable acoustic signatures} - not just random noise
    \item \textbf{Edge contacts create sharper, higher-frequency responses} - geometric effects are real
    \item \textbf{Physics-based features can complement ML features} for better understanding
    \item \textbf{Transfer function analysis reveals the "why" behind acoustic sensing}
\end{itemize}

\section{Summary: What Did We Learn?}

\subsection{The Big Question: Can Acoustic Sensing Work for Robotics?}

\textbf{Answer: YES, but it depends on what you want to detect.}

\subsection{Task Difficulty Ranking}
\begin{enumerate}
    \item \textbf{Material Detection (95\%):} \textcolor{excellent}{READY FOR ROBOTS}
        \begin{itemize}
            \item Wood vs metal creates completely different acoustic signatures
            \item Simple frequency analysis is enough
            \item Any ML algorithm works well
        \end{itemize}
    
    \item \textbf{Binary Edge Detection (90\%):} \textcolor{excellent}{READY FOR ROBOTS}
        \begin{itemize}
            \item Edge vs no-edge has clear temporal differences
            \item Focus on timing dynamics, not just frequency
            \item SVM or Random Forest recommended
        \end{itemize}
    
    \item \textbf{Position Detection (78\%):} \textcolor{good}{NEEDS ENGINEERING}
        \begin{itemize}
            \item Spatial differences are subtle but detectable
            \item Requires balanced feature sets and ensemble methods
            \item Consistent across validation tests
        \end{itemize}
    
    \item \textbf{3-Class Edge Detection (67\%):} \textcolor{poor}{CHALLENGING}
        \begin{itemize}
            \item Contact/Edge/No-contact states overlap significantly
            \item Needs all available features and complex algorithms
            \item Gradient Boosting essential - other methods fail
        \end{itemize}
\end{enumerate}

\subsection{Practical Guidelines for Robot Developers}

\textbf{For Material Discrimination:}
\begin{itemize}
    \item Use spectral features (frequency content)
    \item Any ML algorithm will work - start simple
    \item Expect 95\%+ accuracy reliably
\end{itemize}

\textbf{For Edge Detection:}
\begin{itemize}
    \item Focus on temporal dynamics (timing changes)
    \item Use SVM or Random Forest
    \item Binary edge detection works well, 3-class is hard
\end{itemize}

\textbf{For Position Detection:}
\begin{itemize}
    \item Use balanced feature sets (spectral + temporal)
    \item Ensemble methods required (Random Forest/Gradient Boosting)
    \item 78\% accuracy - good enough for some applications
\end{itemize}

\subsection{What Makes Acoustic Sensing Work?}
\begin{itemize}
    \item \textbf{Different materials have different acoustic resonances}
    \item \textbf{Contact geometry affects acoustic coupling efficiency}
    \item \textbf{Edge interfaces create sharp, high-frequency transients}
    \item \textbf{Spatial changes create subtle but detectable acoustic shifts}
    \item \textbf{The physics is real and measurable}
\end{itemize}

\subsection{Implementation Recommendations}

\textbf{Ready for Deployment:}
\begin{itemize}
    \item Material detection systems
    \item Binary edge detection for manipulation
\end{itemize}

\textbf{Research/Engineering Needed:}
\begin{itemize}
    \item Complex multi-class problems
    \item Noise robustness in real environments
    \item Real-time processing optimization
\end{itemize}

\subsection{Final Assessment}
Acoustic sensing is a \textbf{viable and promising technology} for robotic touch discrimination. It works particularly well for material detection and binary edge detection, with moderate success for position detection. Complex multi-class problems remain challenging but not impossible. The technology is ready for specific applications today, with clear paths for expanding capabilities through continued research.

\end{document}